{\color{textadded}
\chapter{Considerações Finais}
\label{chap:cons}

Este projeto se propôs, como objetivo geral, desenvolver um jogo digital que auxilie o processo de ensino-aprendizagem dos conceitos sobre personas em IHC. Até este momento os objetivos específicos foram alcançados parcialmente (Seção \ref{sec:objetivos}). 

Um dos objetivos é adquirir conhecimento sobre os principais conceitos envolvidos neste trabalho, e isto foi alcançado através da pesquisa e aplicação prática de conceitos como o \textit{Playcentric Design Process}. Foi utilizado as personas definidas no projeto (Apêndice \ref{ap:persona}) em momentos como a de ideação, construção e validação do protótipo do jogo para a representação dos jogadores-alvo. Elas continuarão a serem utilizadas ao decorrer do desenvolvimento do jogo, na qual será validado o jogo funcional durante e após o desenvolvimento do software, como previsto no \textit{Playcentric Design Process}. 

Foram também aplicados conceitos fundamentais da Engenharia de Software como por exemplo: levantamento de requisitos (Seção \ref{ssec:requisitos}), criação de validação de um protótipo de papel (Apêndice \ref{ap:proto_papel}), definição de arquitetura (Seção \ref{sec:arq}), gerência e configuração de software (Seção \ref{sec:gces}) e outros. O desenvolvimento do jogo digital também já foi bem trabalhado através da ideação, construção e validação do protótipo do jogo (Apêndice \ref{ap:proto_papel}), faltando principalmente a fase de produção do código fonte do software, no qual será o foco do TCC2. Todos os objetivos definidos do trabalho podem ser encontrado na Seção \ref{sec:objetivos}.

Para a próxima fase do projeto (TCC2) será executado na prática todo o planejamento e estruturas definidas durante as fases do TCC1.1 e TCC1.2, na qual foram elicitados requisitos e validados com o protótipo de papel, definida a arquitetura e a gerência e configuração do software. Para isto serão utilizados métodos de desenvolvimento de software como o Scrum, XP e Kanban (Seção \ref{sec:mds}). Concluindo, será desenvolvido o software final e validado com as personas e usuários finais, assim como proposto no \textit{Playcentric Design Process}. O cronograma dessa fase é apresentado na Seção \ref{ssec:tcc2}.

}