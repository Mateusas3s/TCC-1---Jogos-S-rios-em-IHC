\chapter{Introdução}
\label{chap:intro}

As Tecnologias de Informação e Comunicação (TICs) conquistam mais espaço e se consolida no cotidiano das pessoas \cite{Sales2020}. Essa interação entre seres humanos e as TICs abrange diversos tipos de perfis de usuários e de contexto de uso, assim é exigido do profissional, que desenvolvem essas tecnologias, certas habilidades. A habilidade de criar esses sistemas, envolve o conhecimento de técnicas, ferramentas e métodos que englobam tanto a Engenharia de Software (ES) quanto a área de Interação Humano-Computador (IHC) \cite[p. 2]{barbosa_silva}.  

Na Engenharia de Software, o campo de IHC preocupa-se em definir como serão as interações entre as ações humanas e os sistemas computacionais, elaborando interfaces para que isso seja possível \cite{queiroz, sommariva} \cite[p. 89,90]{acm_curricula}. O ensino de IHC nos cursos de graduação e pós-graduação na área de Tecnologia da Informação (TI), como é o caso da Engenharia de Software, possui esse objetivo, propiciar a formação de profissionais qualificados capazes de desenvolver interfaces para sistemas computacionais com qualidade, direcionados a atender de forma satisfatória às necessidade e expectativas do usuário \cite[p. 89,  90]{acm_curricula} \cite[p. 7-14]{barbosa_silva}. No entanto os princípios oriundos do campo de IHC podem não ser tão bem aderidos pelos profissionais que desenvolvem software, que acabam por tratar esses conceitos como secundários. Isso se deve a falta de compreensão desses profissionais, que focam mais na parte interna dos sistemas \cite{sommariva}.

O meio acadêmico tem investido no uso de novas abordagens e tecnologias como recurso complementar no processo de ensino-aprendizagem, auxiliando no desenvolvimento de atividades pedagógicas inovadoras e colaborativas em diversas áreas \cite{battistella, brito}, incluindo em IHC \cite{Sales2020,Sales2020UsoTDS}. Os jogos sérios fazem parte dessas abordagens que vêm se tornando cada vez mais populares na educação em computação, pois podem aumentar a eficácia e o engajamento da aprendizagem \cite{battistella, brito, queiroz}.

Esse cenário motivou a criação de jogos educacionais com o objetivo de auxiliar o ensino e aprendizagem de IHC voltado para cursos de graduação e pós-graduação na área de Ciência da Computação. Diversas propostas de jogos para o ensino-aprendizagem de IHC já existem, seja com o intuito de introduzir conceitos iniciais, reforçá-los ou desenvolver habilidades mais práticas \cite{darin}. Dentre estes pode-se citar o UsabilityGame \cite{sommariva}, UsabiliCity \cite{ferreira2014a,ferreira2014b} e MACteaching \cite{brito,queiroz} como também jogos não-digitais como Desafio Goople \cite{darin}, G4H \cite{juca2017} e o G4NHE \cite{deSousa2019} .

Este trabalho visa o desenvolvimento de um jogo que auxilie o processo de ensino e aprendizagem em IHC em curso de graduação e pós-graduação, porém com um conteúdo diferente dos jogos já existentes. O conteúdo disciplinar o qual este jogo trata é a técnica de \textit{Personas}. Conforme é citado por \cite[p. 176]{barbosa_silva}, com base nos autores \citeonline{cooper99, pruitt, cooper07}, \textit{"uma persona é um personagem fictício, arquétipo hipotético de um grupo de usuários reais, criada para descrever um usuário típico"}. 

Em outras palavras, embora fictícia, uma \textit{persona} é definida com um rigor de detalhes de forma que represente bem o público-alvo de usuários reais, para o qual a interface deve ser direcionada  \cite[p. 177]{barbosa_silva}. Sendo assim, essa técnica é uma ferramenta poderosa para a elaboração do design, pois uma vez que ele atenda os objetivos das personas elencadas, o design da interface estará satisfazendo seus usuários reais \cite[p. 77]{cooper99}.

Para um engenheiro de software ter o conhecimento de tal técnica se faz necessário em situações nas quais o público-alvo do sistema engloba muitas pessoas ou que é custosa a presença frequente do usuário durante as etapas do projeto. Desta forma torna-se difícil a validação constante do design, o que pode acarretar num produto que não atenda os objetivos e necessidades do cliente \cite[p. 176]{barbosa_silva}.

{\color{textmodified}
Num cenário de trabalho remoto, no qual existe essa dificuldade de interação com o usuário, a importância do conhecimento desta técnica para o profissional de desenvolvimento de software justifica a escolha do tema do jogo. Neste trabalho é apresentado o PersonaDesignGame, um jogo educacional sobre personas. Ele é um jogo do gênero de perguntas e respostas, onde o jogador progride ao responder corretamente às questões. Ao longo das fases o jogador recebe recompensas, que irão compor algumas personas e ao final, o jogador terá exemplos de personas construídas.
}
\section{Objetivos}
\label{sec:objetivos}

{\color{textadded}
Nesta seção é apresentado os objetivos geral e específicos deste trabalho.
}

\subsection{Objetivo Geral}

O objetivo geral deste trabalho é desenvolver um jogo digital que auxilie o processo de ensino-aprendizagem dos conceitos sobre \textit{personas} em IHC.

\subsection{Objetivo Específico}
\label{ssec:obj_especifico}
\begin{itemize}
    \item \textbf{OE01} - Desenvolver conhecimento sobre os conceitos principais envolvidos neste trabalho;
    \item \textbf{OE02} - Utilizar personas que representem os jogadores-alvo durante o processo de desenvolvimento;
    \item \textbf{OE03} - Aplicar alguns conceitos fundamentais da Engenharia de Software, no processo de design e desenvolvimento de um jogo digital; e    
    \item \textbf{OE04} - Desenvolver o jogo digital para apoio ao ensino e aprendizado de personas;
\end{itemize}

\section{Plano de Trabalho}
\label{sec:plano_trab}
{\color{textmodified}
Primeiramente foi realizado o planejamento identificando os requisitos necessários para a realização do projeto, mapeando as atividades a serem realizadas e construindo um cronograma \cite[p. 74]{Pressman_2000}. Foi tomado por base a própria dinâmica do desenvolvimento do Trabalho de Conclusão de Curso presente no curso de Engenharia de Software, porém sofrendo uma adaptação, no qual o projeto é dividido em três fases: TCC1.1, TCC1.2 e TCC2. 

Esta divisão de TCC1.1 e TCC1.2 foi feita devida a dupla do trabalho cursar a disciplina de TCC1 em semestres distintos. O TCC1.1 foi realizado pelo autor Mateus Augusto e o TCC1.2 pelo autor Rossicler Júnior. Na Figura \ref{Fig:eap2.png} são apresentadas as fases e suas respectivas atividades em forma de uma Estrutura Analítica de Projeto (EAP) \cite[p. 112]{pmbok}. %ES pg 74
}

\newpage

\begin{figure}[htbp]
	\centering
		\includegraphics[angle=90,origin=c,keepaspectratio=true,scale=0.5]{figuras/eap2.png}
	\caption{{\color{textadded}Estrutura Analítica do Projeto - Próprios Autores}}
	\label{Fig:eap2.png}
\end{figure}

{\color{textmodified}
As macro atividades apresentadas na EAP (Figura \ref{Fig:eap2.png}) tem relação com o objetivo deste trabalho. Elas são subdivididas outras vez, evidenciando atividades mais específicas que têm propósito de atingir os seus objetivos específicos. A seguir é relatada a relação entre atividades e objetivos e o cronograma para cada uma das fases do trabalho. 
}
\newpage
\subsection{\textcolor{textmodified}{Fase do TCC 1.1  realizada pelo autor Mateus Augusto}}

{\color{textadded}
A Figura \ref{Fig:cronograma_tcc1_1.png} apresenta o cronograma seguido para a execução das atividades definidas na EAP (Figura \ref{Fig:eap2.png}) na fase do TCC1.1. A seguir são listadas as atividades da EAP fazendo a relação com os objetivos específicos do trabalho (Subseção \ref{ssec:obj_especifico}).
}

\begin{figure}[htbp]
	\centering
		\includegraphics[keepaspectratio=true,scale=0.81]{figuras/cronograma_tcc1.1.png}
	\caption{\textcolor{textmodified}{Cronograma de Trabalho TCC1.1 - Fonte: Mateus Augusto}}
	\label{Fig:cronograma_tcc1_1.png}
\end{figure}

{\color{textmodified}
\begin{itemize}
    \item A macro atividade 'pesquisa científica' é composta pelas atividades de 'planejamento da pesquisa' e 'revisão bibliográfica' tendo como finalidade alcançar o OE01;
    \item A macro atividade 'definição do escopo' é formada pelas atividades do 'survey' e 'elaboração das personas' com foco em alcançar o OE02; e
    \item A macro atividade 'planejamento do jogo' é composta pelas atividades de 'estruturação do processo de desenvolvimento', 'ideação do jogo' e 'elaboração do software' possuindo como objetivo o OE02, o OE03 e o OE04.
\end{itemize}
}
\newpage
\subsection{\textcolor{textmodified}{Fase do TCC 1.2 realizada pelo autor Rossicler Júnior}}
{\color{textadded}

A Figura \ref{Fig:cronograma_tcc1_2.png} apresenta o cronograma seguido para a execução das atividades definidas na EAP (Figura \ref{Fig:eap2.png}) na fase do TCC1.2. A seguir são listadas as atividades da EAP fazendo a relação com os objetivos específicos do trabalho (Subseção \ref{ssec:obj_especifico}).

\begin{figure}[htbp]
	\centering
		\includegraphics[keepaspectratio=true,scale=0.81]{figuras/cronograma_tcc1.2.png}
	\caption{\textcolor{textadded}{Cronograma de Trabalho TCC1.2 - Fonte: Rossicler Júnior}}
	\label{Fig:cronograma_tcc1_2.png}
\end{figure}

\begin{itemize}
    \item A macro atividade 'pesquisa científica' é composta pelas atividades de 'planejamento da pesquisa' e 'revisão bibliográfica' tendo como finalidade alcançar o OE01;
    \item A macro atividade 'protótipo' é formada pelas atividades de 'validação do protótipo de baixa fidelidade' e 'criação do de alta fidelidade' com foco em alcançar o OE03; e
    \item A macro atividade 'planejamento do jogo' é composta pelas atividades de 'possíveis evoluções no jogo', 'definição de arquitetura', 'configuração do ambiente de desenvolvimento' possuindo como objetivo o OE03 e o OE04.
\end{itemize}
}
\newpage
\subsection{\textcolor{textmodified}{Fase do TCC 2 que será realizada por ambos autores}}
\label{ssec:tcc2}

{\color{textadded}
A Figura \ref{Fig:cronograma_tcc2.png} apresenta o cronograma seguido para a execução das atividades definidas na EAP (Figura \ref{Fig:eap2.png}) na fase do TCC2. A seguir são listadas as atividades da EAP fazendo a relação com os objetivos específicos do trabalho (Subseção \ref{ssec:obj_especifico}).
}

\begin{figure}[htbp]
	\centering
		\includegraphics[keepaspectratio=true,scale=0.81]{figuras/cronograma_tcc2.PNG}
	\caption{\textcolor{textmodified}{Cronograma de Trabalho TCC2 - Fonte: Próprios Autores}}
	\label{Fig:cronograma_tcc2.png}
\end{figure}

{\color{textmodified}
\begin{itemize}
    \item A macro atividade 'pesquisa científica' é composta pelas atividades de 'apresentação dos resultados' e 'considerações finais' tendo como objetivo o OE01; e
    \item A macro atividade 'desenvolvimento do jogo' é composta pelas atividades de 'criação e validação do protótipo de alta fidelidade', 'configuração do ambiente de desenvolvimento', 'estudo das tecnologias', 'construção do software' e 'estruturação para contribuição' possuindo como objetivo o OE02, OE03 e OE04.
\end{itemize}
}

{\color{textmodified}
As atividades presentes no EAP (Figura \ref{Fig:eap2.png}) são abordadas ao longo do projeto por meio de passos, etapas e tarefas. Estas estão apresentadas no Capítulo \ref{chap:Metodo}, organizadas em um processo metodológico a fim de serem alcançados os objetivos deste estudo.}

\section{Estrutura do Trabalho}

Este trabalho está estruturado, até o momento, em quatro capítulos. Este, o Capítulo \ref{chap:intro}, mostra uma visão geral dos conceitos que contextualizam este trabalho. Além disso é apresentado seu objetivo, o plano de trabalho e sua estrutura organizacional.

O Capítulo \ref{chap:Metodo}, apresenta a estrutura do processo metodológico utilizado neste trabalho. Nele é relatado a metodologia usada da revisão da bibliográfica até o desenvolvimento do jogo.  

O Capítulo \ref{chap:ref}, apresenta os principais conceitos que envolvem este trabalho e como eles se relacionam. Neste capítulo está descrito uma base sobre IHC, Personas e Jogos.

{\color{textmodified}
O Capítulo \ref{chap:des}, apresenta o progresso do desenvolvimento do projeto. Neste primeiro momento são relatados resultados parciais do design técnico, da ideação do jogo e de definições do software.
}

%O capítulo Apresentação dos Resultados...

%O capítulo Considerações Finais ...