\begin{table}[htbp]
\centering
\caption{Persona Secundária}
\label{tab:Table_persona2}
\small
\begin{tabular}{| m{0.25\textwidth} m{0.65\textwidth}|}
\hline \multicolumn{2}{|c|}{\textbf{Identidade}} \\ \hline
& \\

\begin{center} \includegraphics[scale=0.06]{figuras/personas/model-2911332_1920.jpg} \end{center} 

&

\textbf{Nome: } Afonso Souza de Queiroz

\textbf{Idade:} 19 anos

\textbf{Ocupação:} Estudante de Engenharia de Software na UnB, Faculdade do Gama

\\ \hline


\multicolumn{2}{|c|}{\textbf{Descrição}} \\ \hline
\multicolumn{2}{|p{15cm}|}{
        Já joguei alguns jogos educacionais cujo meu principal objetivo era aprender um conteúdo novo. Além disso era interessante quando o jogo possibilitava que eu revisasse o conteúdo e avaliasse o que tinha aprendido. Eu jogava com certa moderação, pois vejo esse tipo de jogo apenas como uma ferramenta que pode me auxiliar no processo de aprendizagem. Por mais que atualmente eu não esteja usando algum jogo educacional, me senti satisfeito em alcançar meus objetivos de estudo ao jogar e me sentiria da mesma forma se houvesse algum jogo que me auxiliasse no processo de aprendizagem nas matérias da faculdade. 
        
        Não fiz ainda a disciplina de IHC, por isso não tenho muito conhecimento em relação a elaboração de design de interfaces, sendo que o pouco que tenho veio de projetos de outras disciplinas e atividades extra curriculares.
        
        Quando estudo opto por realizar pesquisas na internet. Em alguns casos assisto vídeo aulas e também  utilizo do material disponibilizado pelo professor. Outros meios que uso para aprender o conteúdo são, estudar com meus próprios colegas da disciplina e tirar dúvidas com colegas que já fizeram a disciplina.Os requisitos do jogo que me levariam a usá-lo são: 
        
        \begin{itemize}
            \item um design atraente e consistente; 
            \item o jogo ser fácil de aprender a jogar e fácil de se jogar, com regras claras; 
            \item um bom uso de fontes e cores; 
            \item que o jogo ofereçam feedbacks relevantes;
            \item pontos e recompensas.
        \end{itemize}
        
        Ao usar um jogo eu espero aprender o conteúdo ali apresentado, espero que seja algo desafiador, divertido e satisfatório aprender jogando. Jogos que prendem minha atenção e me envolvem para  aprender o conteúdo são bem relevantes, ainda mais se eu conseguir perceber a importância do conteúdo que estou aprendendo. 
        } \\ \hline
\end{tabular}
\end{table}