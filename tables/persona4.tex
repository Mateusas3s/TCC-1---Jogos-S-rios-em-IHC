

\begin{table}[htbp]
\centering
\caption{Anti Persona}
\label{tab:Table_persona4}
\small
\begin{tabular}{| m{0.25\textwidth} m{0.65\textwidth}|}
\hline \multicolumn{2}{|c|}{\textbf{Identidade}} \\ \hline
& \\

\begin{center} \includegraphics[scale=0.06]{figuras/personas/man-1209494_1920.jpg} \end{center} 

&

\textbf{Nome: }  Rafael Medeiros

\textbf{Idade:} 28 anos

\textbf{Ocupação:} Estudante de Educação Física na UnB, campus Darcy Ribeiro.

\\ \hline


\multicolumn{2}{|c|}{\textbf{Descrição}} \\ \hline
\multicolumn{2}{|p{15cm}|}{
        Não tenho o costume de usar jogos educacionais, devo ter jogado um alguma vez na vida, mas não me interessei muito e rapidamente o jogo se tornou monótono. Gosto de jogar futebol e basquetebol. Em relação ao conhecimento acadêmico, me concentro apenas nas disciplinas do meu curso.
        
        Quando vou estudar ou sanar alguma dúvida que tenho sobre o conteúdo eu utilizo todos os recursos possíveis, priorizando o que for mais prático e objetivo. Geralmente eu tiro dúvidas com colegas que já fizeram a disciplina e colegas que cursam a disciplina comigo. Somente se realmente necessário que vou tirar as dúvidas com os monitores ou professor da matéria. Os requisitos do jogo que me levariam a usá-lo são: 
        
        \begin{itemize}
            \item um design atraente e consistente; 
            \item o jogo ser fácil de aprender a jogar e fácil de se jogar, com regras claras; 
            \item um bom uso de fontes e cores; 
            \item  o jogo ter uma história que me envolvesse seria algo interessante;
            \item que o jogo ofereçam feedbacks relevantes;
            \item pontos e recompensas;
            \item gosto de jogos que promovam interação com outros jogadores, como por exemplo ranking dos jogadores.
        \end{itemize}
        
        Ao usar um jogo eu espero aprender o conteúdo ali apresentado, espero que seja algo desafiador, divertido e satisfatório aprender jogando. Jogos que prendem minha atenção e me envolvem para  aprender o conteúdo são bem relevantes, ainda mais se eu conseguir perceber a importância do conteúdo que estou aprendendo.

       } \\ \hline
\end{tabular}
\end{table}