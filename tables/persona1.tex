\begin{table}[htbp]
\centering
\caption{Persona Primária}
\label{tab:Table_persona1}
\small
\begin{tabular}{| m{0.25\textwidth} m{0.65\textwidth}|}
\hline \multicolumn{2}{|c|}{\textbf{Identidade}} \\ \hline
& \\

\begin{center} \includegraphics[scale=0.06]{figuras/personas/portrait-3353699_1920.jpg} \end{center} 

&

\textbf{Nome: } Victor Matheus Farias

\textbf{Idade:} 19 anos

\textbf{Ocupação:} Estudante de Engenharia de Software na UnB, Faculdade do Gama.

\\ \hline


\multicolumn{2}{|c|}{\textbf{Descrição}} \\ \hline
\multicolumn{2}{|p{15cm}|}{
        Aprender algum conteúdo novo é meu o principal objetivo ao usar jogos educacionais. Atualmente eu uso alguns jogos educacionais, mas com uma frequência moderada. Vejo-os como ferramentas que me auxiliam no processo de aprendizagem. Estou cursando a disciplina de IHC e não tenho muito conhecimento em relação a elaboração de design de interfaces e o pouco que tenho está somente no âmbito disciplinar, daquilo que já vi na disciplina. Desejaria utilizar um jogo que me ajudasse a aprender o conteúdo.
       
        Geralmente quando vou estudar ou sanar alguma dúvida que tenho sobre o conteúdo eu recorro primeiramente à artigos da internet. Em alguns casos assisto vídeo aulas e também utilizo do material disponibilizado pelo professor.
        
        Outros meios que uso para sanar as dúvidas que tenho em relação ao conteúdo é perguntar a colegas que já fizeram a disciplina e também geralmente estudo com colegas que estão fazendo a disciplina comigo. Busco em alguns casos tirar as dúvidas com os monitores ou professor.
        Os requisitos do jogo que me levam a usá-lo são: 
        \begin{itemize}
            \item um design atraente e consistente; 
            \item o jogo ser fácil de aprender a jogar e fácil de se jogar, com regras claras; 
            \item um bom uso de fontes e cores; 
            \item que o jogo ofereçam feedbacks relevantes. 
        \end{itemize}
        
        Ao usar um jogo eu espero ter certeza que vou  aprender o conteúdo ali apresentado, espero que seja algo desafiador, divertido e satisfatório aprender jogando. 
        
        Jogos que prendem minha atenção e me envolvem para aprender o conteúdo são bem relevantes, ainda mais se eu conseguir perceber a importância do conteúdo que estou aprendendo.} \\ \hline
\end{tabular}
\end{table}