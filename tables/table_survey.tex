%longtable

\begin{table}[ht]
\centering
\caption{Rastreabilidade do Questionário}
\label{tab:Table_survey}
\begin{tabular}{|l|l|l|}
\hline
\textbf{ID} & \textbf{Objetivos}                                              & \textbf{Origem}   \\ \hline
I01         & - Usar como base para caracterização das personas               & UG, HC            \\ \hline
I02         & - Usar como base para caracterização das personas               & UG, HC            \\ \hline
I03         & - Usar como base para caracterização das personas               & UG, HC            \\ \hline
I04         & \begin{tabular}[c]{@{}l@{}}
              - Usar como base para caracterização das personas.\\ 
              - Controlar se respondente se enquadra no perfil do público-alvo
              \end{tabular}                                                   & UG, HC            \\ \hline
O01         & \begin{tabular}[c]{@{}l@{}}
              - Auxiliar na identificação dos objetivos das personas \\ 
              - Identificar o nível de conhecimento das personas
              \end{tabular}                                                   & UG, HC            \\ \hline
H01         & \begin{tabular}[c]{@{}l@{}}- Identificar o nível de conhecimento das personas \\ 
              - identificar as habilidades das personas
              \end{tabular}                                                   & UG, HC            \\ \hline
H02         & \begin{tabular}[c]{@{}l@{}}
              - Identificar o nível de conhecimento das personas \\ 
              - identificar as habilidades das personas
              \end{tabular}                                                 & UG, HC              \\ \hline
H03         & \begin{tabular}[c]{@{}l@{}} 
              - Auxiliar na validação da oportunidade de intervenção proposta\\ neste trabalho.
              \end{tabular} & UG, HC                                                              \\ \hline
RL01        & - Identificar com quem a persona se relaciona                   & HC                \\ \hline
O02         & \begin{tabular}[c]{@{}l@{}} 
              - Auxiliar na identificação dos objetivos das personas   \\       
              - Auxiliar na validação da oportunidade de intervenção proposta\\ neste trabalho.
              \end{tabular} & UG, HC            \\ \hline
O02.1.1     & - Identificar objetivos das personas                            & UG, HC            \\ \hline
O02.1.2     & \begin{tabular}[c]{@{}l@{}}
              - Auxiliar na identificação dos objetivos das personas\\ 
              - Identificar característica da tarefa realizada pela persona
              \end{tabular}                                                   & UG, HC            \\ \hline
O02.1.3     & - Usar como base para levantamento de alguns requisitos         &   -               \\ \hline
O02.1.4     & - Auxiliar na identificação dos objetivos das personas          & UG, HC            \\ \hline
O02.2.1     & - Identificar objetivos das personas                            & UG, HC            \\ \hline
O02.2.2     & \begin{tabular}[c]{@{}l@{}}
              - Auxiliar na identificação dos objetivos das personas\\ 
              - Identificar característica da tarefa realizada pela persona
              \end{tabular}                                                   & UG, HC            \\ \hline
O02.2.3     & - Usar como base para levantamento de alguns requisitos         &   -               \\ \hline
O02.3.1     & - Identificar objetivos das personas                            & UG, HC            \\ \hline
O02.3.2     & - Auxiliar na identificação dos objetivos das personas                    & UG, HC  \\ \hline
O02.4.1     & - Auxiliar na identificação dos objetivos das personas                    & UG, HC  \\ \hline
RE01        & - Identificar requisitos para o jogo                                      & \begin{tabular}[c]{@{}l@{}}
                                                                                          UG, HC, \\ 
                                                                                          JS, ME
                                                                                         \end{tabular}  \\ \hline
E01         & \begin{tabular}[c]{@{}l@{}}
              - Identificar requisitos para o jogo\\ 
              - Definir as metas de experiência do jogador\end{tabular}                & \begin{tabular}[c]{@{}l@{}}
                                                                                          UG, HC, \\ 
                                                                                          JS, ME
                                                                                         \end{tabular}  \\ \hline
\multicolumn{3}{p{15cm}}{\textbf{Legenda:} UG - \citeonline{usability2020}; HC - \citeonline{barbosa_silva} ; JS - \citeonline{deSales_SousaeSilva_2020}; ME - \citeonline{Petri_Wangenheim_2019}} \\
\end{tabular}
\end{table}