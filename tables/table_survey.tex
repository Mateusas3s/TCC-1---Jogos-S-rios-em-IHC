%longtable

\begin{table}[htbp]
\centering
\caption{Rastreabilidade do Questionário}
\label{tab:Table_survey}
\begin{tabular}{|l|l|l|l|}
\hline
\textbf{ID} & \textbf{Objetivo} & \textbf{Figura} & \textbf{Artigo} \\ \hline
I01     & - Identificar dados demográficos do usuário                                                                                                                                               &  \ref{Fig:survey_pt2.png} & UG, HC \\ \hline
I02     & - Identificar dados demográficos do usuário                                                                                                                                               &  \ref{Fig:survey_pt2.png} & UG, HC \\ \hline
I03     & - Identificar dados demográficos do usuário                                                                                                                                               &  \ref{Fig:survey_pt2.png} & UG, HC \\ \hline
I04     & \begin{tabular}[c]{@{}l@{}}- Identificar dados demográficos do usuário\\ - Variável de controle para identificar perfil\\ segundo o público-alvo\end{tabular}                             &  \ref{Fig:survey_pt2.png} & UG, HC \\ \hline
O01     & \begin{tabular}[c]{@{}l@{}}- Auxiliar na identificação dos objetivos do usuário\\ - Identificar o nível de conhecimento do usuário\end{tabular}                                         &  \ref{Fig:survey_pt3.png} & UG, HC \\ \hline
H01     & \begin{tabular}[c]{@{}l@{}}- Identificar o nível de conhecimento do usuário\\ - identificar as habilidades do usuário\end{tabular}                                                      &  \ref{Fig:survey_pt3.png} & UG, HC \\ \hline
H02     & \begin{tabular}[c]{@{}l@{}}- Identificar o nível de conhecimento do usuário\\ - identificar as habilidades do usuário\end{tabular}                                                      &  \ref{Fig:survey_pt4.png} & UG, HC \\ \hline
H03     & \begin{tabular}[c]{@{}l@{}}- Identificar o nível de conhecimento do usuário\\ - identificar as habilidades do usuário\\ - Auxiliar na validação do objetivo deste trabalho\end{tabular} &  \ref{Fig:survey_pt4.png} & UG, HC \\ \hline
RL01    & - Identificar relacionamentos do usuário                                                                                                                                                &  \ref{Fig:survey_pt5.png} & HC     \\ \hline
O02     & - Auxiliar na identificação dos objetivos do usuário                                                                                                                                    &  \ref{Fig:survey_pt5.png} & UG, HC \\ \hline
O02.1.1 & - Identificar objetivos do usuário                                                                                                                                                      &  \ref{Fig:survey_pt6.png} & UG, HC \\ \hline
O02.1.2 & \begin{tabular}[c]{@{}l@{}}- Auxiliar na identificação dos objetivos do\\ usuário\\ - Identificar característica da tarefa realizada pelo \\usuário\end{tabular}                          &  \ref{Fig:survey_pt6.png} & UG, HC \\ \hline
O02.1.3 & - Usar como base para levantamento de requisitos                                                                                                                                        &  \ref{Fig:survey_pt6.png} & -      \\ \hline
O02.1.4 & - Auxiliar na identificação dos objetivos do usuário                                                                                                                                    &  \ref{Fig:survey_pt6.png} & UG, HC \\ \hline
O02.2.1 & - Identificar objetivos do usuário                                                                                                                                                      &  \ref{Fig:survey_pt7.png} & UG, HC \\ \hline
O02.2.2 & \begin{tabular}[c]{@{}l@{}}- Auxiliar na identificação dos objetivos do usuário\\ - Identificar característica da tarefa realizada pelo \\perfil de usuário\end{tabular}                &  \ref{Fig:survey_pt7.png} & UG, HC \\ \hline
O02.2.3 & - Usar como base para levantamento de requisitos                                                                                                                                        &  \ref{Fig:survey_pt7.png} & -      \\ \hline
O02.3.1 & - Identificar objetivos do usuário                                                                                                                                                      &  \ref{Fig:survey_pt8.png} & UG, HC \\ \hline
O02.3.2 & - Auxiliar na identificação dos objetivos usuário                                                                                                                                       &  \ref{Fig:survey_pt8.png} & UG, HC \\ \hline
O02.4.1 & - Auxiliar na identificação dos objetivos do usuário                                                                                                                                    &  \ref{Fig:survey_pt8.png} & UG, HC \\ \hline
RE01    & \begin{tabular}[c]{@{}l@{}}- Identificar diretrizes de qualidade para o jogo\\ - Identificar requisitos para o jogo\end{tabular}                                                        &  \ref{Fig:survey_pt9.png} & \begin{tabular}[c]{@{}l@{}}UG, HC, \\ JS, ME\end{tabular}       \\ \hline
E01     & \begin{tabular}[c]{@{}l@{}}- Identificar diretrizes de qualidade para o jogo\\ - Identificar requisitos para o jogo\\ - Definir as metas de experiência do jogador\end{tabular}         &  \ref{Fig:survey_pt10.png}& \begin{tabular}[c]{@{}l@{}}UG, HC, \\ JS, ME\end{tabular}       \\ \hline

\multicolumn{4}{p{15cm}}{\textbf{Legenda:} UG - \citeonline{usability2020}; HC - \citeonline{barbosa_silva} ; JS - 
\citeonline{deSales_SousaeSilva_2020}; ME - \citeonline{Petri_Wangenheim_2019}} \\
\end{tabular}
\end{table}