
\begin{table}[htbp]
\centering
\caption{Persona Suplementar}
\label{tab:Table_persona3}
\small
\begin{tabular}{| m{0.25\textwidth} m{0.65\textwidth}|}
\hline \multicolumn{2}{|c|}{\textbf{Identidade}} \\ \hline
& \\

\begin{center} \includegraphics[scale=0.06]{figuras/personas/girl-919048_1920.jpg} \end{center} 

&

\textbf{Nome: }  Natália Figueiredo

\textbf{Idade:} 23 anos

\textbf{Ocupação:} Estudante de Engenharia de Software na UnB, Faculdade do Gama.

\\ \hline


\multicolumn{2}{|c|}{\textbf{Descrição}} \\ \hline
\multicolumn{2}{|p{15cm}|}{
        Nunca joguei jogos educacionais, pois não conheci nenhum que tinha o propósito de ensinar o que eu desejava. No caso se eu encontrasse um jogo onde eu pudesse aprender um conteúdo novo e pudesse revisá-lo quando necessário seria interessante. Estou cursando a disciplina de IHC e desejaria utilizar um jogo que me ajudasse a aprender o conteúdo, pois não tenho muito conhecimento em relação a elaboração de design de interfaces e o pouco que tenho foi somente a partir da disciplina de IHC.
        
        Geralmente quando vou estudar ou sanar alguma dúvida que tenho sobre o conteúdo eu pesquiso na internet. Em ocasiões bem específicas eu assisto vídeo aulas e também utilizo do material disponibilizado pelo professor. Também tiro dúvidas com colegas que já fizeram a disciplina e colegas que cursam a disciplina comigo. Raras são os casos em que vou tirar as dúvidas com os monitores ou professor. Os requisitos do jogo que me levariam a usá-lo são: 
        
        \begin{itemize}
            \item um design atraente e consistente; 
            \item o jogo ser fácil de aprender a jogar e fácil de se jogar, com regras claras; 
            \item um bom uso de fontes e cores; 
            \item que o jogo ofereçam feedbacks relevantes;
            \item pontos e recompensas.
        \end{itemize}
        
        Ao usar um jogo eu espero aprender o conteúdo ali apresentado, espero que seja algo desafiador, divertido e satisfatório aprender jogando. Jogos que prendem minha atenção e me envolvem para  aprender o conteúdo são bem relevantes, ainda mais se eu conseguir perceber a importância do conteúdo que estou aprendendo.
    
        
       } \\ \hline
\end{tabular}
\end{table}