\begin{table}[htbp]
\centering
\caption{Metas de Experiência do Jogador (\textit{Player Experience})}
\label{tab:Table_px}
\begin{tabular}{|l|l|l|}
\hline
ID   & Nome                   & Descrição                                                                                                                                                                                                      \\ \hline
PX01 & Confiança              & \begin{tabular}[c]{@{}l@{}}O conteúdo e estrutura do jogo devem trazer \\confiança ao usuário de que o jogo irá auxiliá-lo \\no aprendizado do conteúdo de personas\end{tabular}                                                                                                        \\ \hline
PX02 & Desafio                & \begin{tabular}[c]{@{}l@{}}O jogo deve apresentar elementos desafiadores \\ variados que estimulam o jogador\end{tabular}                                                                                            \\ \hline
PX03 & Satisfação             & \begin{tabular}[c]{@{}l@{}}O jogo deve gerar um sentimento de realização \\ pelos resultados alcançado através do \\ desempenho do jogador no progresso do \\ jogo e no aprendizado\end{tabular}                     \\ \hline
PX04 & Diversão               & \begin{tabular}[c]{@{}l@{}}O jogo deve trazer elementos lúdicos que fazem \\ o jogador se sentir bem\end{tabular}                                                                                                      \\ \hline
PX05 & Atenção Focada         & \begin{tabular}[c]{@{}l@{}}O jogo deve prender a atenção do jogador e o \\ envolve em suas atividades\end{tabular}                                                                                                   \\ \hline
PX06 & Aprendizagem Percebida & \begin{tabular}[c]{@{}l@{}}O jogo deve auxiliar o jogador a percebe o quão \\ importante foi o aprendizado do conteúdo e o \\quanto o jogo se destaca em relação a outros \\recurso pedagógicos\end{tabular} \\ \hline
\end{tabular}
\end{table}