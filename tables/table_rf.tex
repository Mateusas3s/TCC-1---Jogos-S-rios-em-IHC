\begin{table}[htbp]
    \centering
\caption{Requisitos Funcionais}
\label{tab:Table_rf}
\begin{tabular}{|l|l|l|}
\hline
ID   & Nome                                                                          & Descrição                                                                                                                                                                                                                \\ \hline
RF01 & \begin{tabular}[c]{@{}l@{}}Conteúdo didático\end{tabular} & \begin{tabular}[c]{@{}l@{}}O jogo deve dispor do conteúdo sobre personas \\em forma de resumos e dicas nas questões.\end{tabular} \\ \hline

RF02 & Dinâmica das questões        & \begin{tabular}[c]{@{}l@{}}As questões devem conter o texto da pergunta e \\alternativas de múltipla escolha ou de verdadeiro \\ou falso. Deve ser possível confirmar a resposta \\selecionada antes de enviá-la para validação.\end{tabular}  \\ \hline

RF03 & Resposta ao Usuário       & \begin{tabular}[c]{@{}l@{}}O jogo deve fornecer feedbacks ao usuário de acerto \\ou erro, após a validação da resposta enviada. \\Caso o jogador tenha errado, o jogo deve \\apresentar a resposta correta, caso tenha acertado \\deve apresentar uma mensagem de congratulação ou \\uma recompensa. Ao selecionar uma etapa da fase do \\jogo o jogador deve confirmar se ele deseja iniciá-la. \end{tabular} \\ \hline   

RF04 & \begin{tabular}[c]{@{}l@{}}Fluxo da partida\end{tabular}            & \begin{tabular}[c]{@{}l@{}}O jogo deve manter bloqueada uma etapa da fase até \\que a etapa anterior seja concluída. Assim que \\concluída a etapa seguinte é desbloqueada. O jogo \\deve permitir o usuário iniciar uma partida \\selecionando qual etapa da fase ele deseja jogar. Deve \\salvar o progresso do usuário. Deve permitir o \\usuário progredir ao responder às questões ao \\selecionar uma alternativa de resposta. O jogo deve \\validar ao resposta selecionada como certa ou errada.\\ E deve permitir o usuário sair da partida do jogo.\end{tabular} \\ \hline    

RF05 & \begin{tabular}[c]{@{}l@{}}Manter o perfil do \\ jogador\end{tabular}             & \begin{tabular}[c]{@{}l@{}}O jogo deve permitir o usuário criar, alterar, \\ visualizar e deletar os seu dados de perfil\end{tabular}\\ \hline

RF06 & Acesso ao jogo                                                               & \begin{tabular}[c]{@{}l@{}}Deve ser possível fazer login, se cadastrar e \\ recuperar a senha para ser acessado o jogo e \\ também fazer logout. Deve ser possível o jogador \\acessar o jogo sem fazer login, porém o progresso \\do jogo é mantido durante a sessão. \end{tabular}\\ \hline

RF07 & Completar Personas                                                            & \begin{tabular}[c]{@{}l@{}}O jogo deve armazenar as características das \\ personas, dos cards coletados durante as partidas \\ do jogo. E deve ser possível visualizá-las em \\ suas respectivas personas.\end{tabular} \\ \hline
\end{tabular}
\end{table}