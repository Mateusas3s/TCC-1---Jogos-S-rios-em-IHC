\begin{resumo}

Atualmente é requerido do profissional que desenvolve software a capacidade e competência de criar sistemas computacionais interativos com qualidade, segundo a visão de Interação Humano-Computador (IHC). Dessa forma, a academia busca abordagens inovadoras e atrativas que venham complementar o ensino e aprendizagem da disciplina de IHC em cursos de graduação e pós-graduação da área de Ciência da Computação. Uma das abordagens que tem sido foco na educação em computação são os jogos sérios, que têm características que ajudam a melhorar a eficácia e engajamento dos alunos no processo de aprendizagem. A literatura mostra alguns jogos aplicados ao ensino em IHC, porém não foi encontrado um que abranja todo conteúdo de IHC. Assim, o objetivo deste trabalho é desenvolver um jogo que auxilie no processo de ensino-aprendizagem em IHC e ajude o aluno no entendimento do conceito e aplicação de \textit{personas}, o PersonaDesignGame (PDG). Este trabalho está sendo desenvolvido nas seguintes fases: fase de Revisão Bibliográfica, dividida em duas etapas (Revisão Não Sistemática e Revisão Sistemática da Literatura); a fase de Definição do Escopo, envolvendo um \textit{Survey} e a técnica de Personas; e a fase de Desenvolvimento do Jogo, que abrange todo o processo de design de jogos \textit{Playcentric Design Game} e aspectos da Engenharia de Software. Os resultados parciais deste trabalho são as personas elaboradas, os requisitos elicitados e um esboço inicial do design do jogo. Para a segunda fase do trabalho tem-se como objetivo validar e refinar a ideia do jogo, construí-lo e avaliá-lo.

% Colocar resumidamente o que será feito em TCC2

%  O resumo deve ressaltar o objetivo, o método, os resultados e as conclusões 
%  do documento. A ordem e a extensão
%  destes itens dependem do tipo de resumo (informativo ou indicativo) e do
%  tratamento que cada item recebe no documento original. O resumo deve ser
%  precedido da referência do documento, com exceção do resumo inserido no
%  próprio documento. (\ldots) As palavras-chave devem figurar logo abaixo do
%  resumo, antecedidas da expressão Palavras-chave:, separadas entre si por
%  ponto e finalizadas também por ponto. O texto pode conter no mínimo 150 e 
%  no máximo 500 palavras, é aconselhável que sejam utilizadas 200 palavras. 
%  E não se separa o texto do resumo em parágrafos.

 \vspace{\onelineskip}
    
 \noindent
 \textbf{Palavras-chaves}: Personas, Jogos, Interação Humano-Computador.
\end{resumo}
